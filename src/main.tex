% arara: xelatex: { shell: yes }
% arara: biber
% arara: nomencl
% arara: xelatex: { shell: yes }
% arara: xelatex: { shell: yes }
\documentclass[12pt, a4paper, english]{ttlab-qualify}
% mgliche Optionen:
% - ngerman
% - english
% - minted
% - algorithm
% - nomencl
% - nolibertine

% P A C K A G E S
\usepackage[onehalfspacing]{setspace}
\usepackage{acronym}
\usepackage[super]{nth}
\usepackage{framed, enumitem}
\usepackage{textcomp}
\usepackage{hyperref} %[hidelinks] option for final compile
\usepackage[edges]{forest}
\usepackage{tikz}

% Class scrreprt Warning fix
\usepackage{scrhack}

% Continuous figure numbering
\counterwithout{figure}{chapter}
\counterwithout{table}{chapter}
\counterwithout{algorithm}{chapter}



\usepackage{gb4e} %interlinear glossing
\nosinglegloss % remove single spacing for glossing
\usepackage{algorithm}
\usepackage{algpseudocode}
\usepackage{algorithmicx}
\usepackage{amsmath}
\usepackage{amsfonts}
\usepackage{stmaryrd}

\newcommand{\algorithmautorefname}{Algorithm} % full algorithm references

\KOMAoptions{listof=totoc}


\addbibresource{main.bib}

%\includeonly{02overview.tex} % input control

\begin{document}
    \titlehead{
        Ricardo Lukas Jung\\
        6227492\\
        Empirische Sprachwissenschaft (B.A.)\\
        Phonetik \& Digital Humanities \\
        15\textsuperscript{th} Semester\\
        s2458588@stud.uni-frankfurt.de
    }
    %\subject{Thesis submitted in fulfilment of the requirements for the degree of Bachelor of Arts}
    \subject{Bachelor Thesis}
    \author{Ricardo Lukas Jung}
    \title{Lexicalizing a BERT Tokenizer}
    \subtitle{Building Open-End MLM for Morpho-Syntactically Similar Languages}
    \date{Date of Submission: \\\today}
    \publishers{Text Technology Lab\\Prof. Dr. Alexander Mehler\\Dr. Zakharia Pourtskhvanidze}

    \maketitle

    \cleardoublepage
    \thispagestyle{empty}
    I thank my supervisors Prof. Dr. Alexander Mehler and Dr. Zakharia Pourtskhvanidze for their competence and professionalism, without whom this thesis across universitary departments would not have happened.
    My explicit thanks go to Manuel Stoeckel and Guiseppe Abrami for their valuable patience, advice and encouragement.
    Thanks to my close ones for your trust and support.

    \chapter*{Abstract}
    \thispagestyle{empty}
    This is the abstract: what is this about? what was done? what where the results?



    \cleardoubleoddpage
    \pagenumbering{Roman}

    \tableofcontents
    \thispagestyle{empty}
    \newpage
    \listoffigures
    \setcounter{page}{1}
    \newpage
    \listoftables
    \newpage

%    \newcommand{\sectionnumbering}[1]{%
%        \setcounter{section}{0}%
%        \renewcommand{\thesection}{\csname #1\endcsname{section}}%
%    }

    \chapter*{List of Acronyms}
    \addcontentsline{toc}{chapter}{List of Acronyms}
    \begin{acronym}
        \acro{bbgc}[bbgc]{bert-base-german-cased}
        \acro{bert}[BERT]{Bidirectional Encoders from Transformers}
        \acro{bpe}[BPE]{Byte Pair Encoding}
        \acro{bwe}[BWE]{Bert WordPiece}
        \acro{cl}[CL]{Computational Linguistics}
        \acro{gerparcor}[GerParCor]{German Parliamentary Corpus}
        \acro{gpt}[GPT]{Genertive Pretrained Transformers}
        \acro{hanta}[HanTa]{Hanover Tagger}
        \acro{lm}[LM]{language model}
        \acro{lstm}[LSTM]{Long Short-Term Memory}
        \acro{ml}[ML]{Machine Learning}
        \acro{mlm}[MLM]{Masked Language Model}
        \acro{nlp}[NLP]{Natural Language Processing}
        \acro{pos}[POS]{Part of Speech}
        \acro{ttl}[TTLab]{Text Technology Lab}
        \acro{wm}[WM]{Wordmap}

    \end{acronym}

    \cleardoubleoddpage
    %\pagenumbering{arabic}
    %\sectionnumbering{arabic}

    % BEGINNING OF BODY

    \chapter{Introduction}
    \pagenumbering{arabic}
    \setcounter{page}{1}
    \label{ch:introduction}
    
This thesis showcases the use of specific intervention in tokenization subsystems of machine learning.
The intent of this thesis is to inject linguistic bias into the machine learning framework of BERT to sharpen the analytical capacities of a masked language model.
In this chapter the background, intentions and scope of the thesis are covered.


\section{Motivation}
\label{sec:motivation}
\uppercase{Why is this subject relevant}
There is an ongoing urge in the \acf{cl} community to understand natural language.
Research in the past decades shows use of frequentist and statistical methods (such as \uppercase{ZITATION}) to their advantage, leading to the emergence of the first machine learning (ML) models.
It became apparent that these ML models are the best currently available approach to an automated understanding of natural language.
The structural parallels of machine learning to human learning have often been drawn (\uppercase{zitation)}) to demonstrate how similar and more importantly: how different both can be.
A powerful feature of \ac{ml} (as opposed to human learning) is the possibility of actively controlling the the learning parameters in a supervised environment.
To test the efficiency of \uppercase{ML} parameters a variety of tasks (\uppercase{zitation}) are designed and applied.
A trained model will yield performance scores based on the quality of its training, much like humans on language tests.
But the automated modeling of language is not the first instance language modelling in a broader sense.
Traditional linguistics (\uppercase{definition} has produced fundamental research the prior to the discovery of \uppercase{ML} architectures and their implementation.
While generic ML frameworks seem appealing in the presumption that they require less work to reach somewhat satisfactoy results,
they are far from complete or perfect.
The integration of aforementioned traditional linguistic knowledge into learning processes for machine learning is the underlying motivation of this thesis.
\textbf{flota} \cite{FLOTA}
Language learners usually build up a lexicon consisting of lexemes which they will have to analyze accurately in order to be productive in that target language.
A ML model relies on a tokenizer to create such a vocabulary  (\textbf{ZITATION}).
It is programmed to segment tokens into subwords (if possible) and provide a vocabulary comprising all the components needed to analyze a given string.
Ideally those subwords will be part of the functional vocabulary in the target language, so called morphemes \textbf{ERKLÄRUNG}.
A morpheme is defined as the smallest unit carrying meaning in a language.
The morphemes of a language and its generated tokenizer vocabulary rarely coincide.
Typically, tokanizer vocabularies will contain a lot of noise and linguistically nonsensical segmentations or words.
Following the guiding principle that \textbf{input quality is ouput quality} not only in language learning, the morpheme vocabulary is identified as the point of leverage in the upcoming section.
Note: explain why i use tokens and words, they are interchangeable right?
holistic, need less attention to produce satisfactory
\uppercase{not just to push F, but to find a viable method of morphemic tokenization}


\section{Hypotheses}
\label{sec:hypothesis}

The following research questions will be formulated for testing:
\begin{framed}
    \begin{itemize}[itemindent=1em]
        \item[HYP1:] Adjustments to tokenization have significant impact the performance of a language model.
    \end{itemize}
\end{framed}
How to achieve this hypothesis?
\begin{framed}
    \begin{itemize}[itemindent=1em]
        \item[HYP2:] Providing lexical information to a tokenizer increases benchmark accuracy on MLM tasks.
    \end{itemize}
\end{framed}
How to achieve this hypothesis?

\section{Scope and Structure}
\label{sec:scope-and-structure}

The following chapters are sorted into three parts.
To outline the research domain, a brief summary of the current state of morphological language modeling is given.
Next, german is described paying special attention to its morphological complexity and peer languages.
This serves as preface to the methodology, connecting characteristically matching languages to form a pool of possible  target languages.

As main part of this thesis, the methodology is layed out.
It is sectioned into a theoretical part which focuses on what implements are used and the value they hold towards lexicalizing a tokenizer

What is covered and what not?
What is the shape of this thesis and what order does it have?


    \chapter{Overview}
    \label{ch:overview}
    %define morpheme vs subword
%define morphological tokenization

Morphological tokenization can be understood as the process of identifying segments in text that are a productive in a given language, carrying meaning and hence also fitting the definition of a morpheme.
describe the most recent findings on morphologically pretrained models in machine learning literature

%findings on POS effect on ML

\section{Related Works}
\label{sec:related-works}

In the past, many efforts towards morphological tokenization have been made.
This thesis was mainly inspired by the FLOTA

Earlier generalized attempts like morfessor (\cite{morfessor}) have been outperformed by Sequence based models that also use linguistic morphology ~\cite{subwordvsmorfessor} .
Notably, top-down generation of subword vocabularies has shown promising results for tokenization in fusional languages.
This aligns with the notion that standard BPE (\cite{BPE}) or WordPiece ~\cite{WORDPIECEGOOGLE} tokenization effectivity suffers from complex morphology causing a big vocabulary.
The overall comparison ~\cite[134]{subwordvsmorfessor} shows an increase in performance for languages of similar morphological complexity.
It is interesting to see that this form of tokenization performs less well for English, a language that has seen a decline in morphology.
Much better benchmarks are reached applying  its agglutinative fusional peers, e.g.\ Italian, Latin, Spanish, Russian
\citeauthor{TOKENIZATIONIMPACT} find that the vocabulary size plays a special role in morphological tokenization and even define a ratio vocabulary size ratio between 20 to 40\% to the number of model parameters depending on the type of tokenizer (~\cite[11--12]{TOKENIZATIONIMPACT}).
Since tokenization and vocabulary are obviously interdependent, the vast amount of typological variety seen in languages raises the question: is there a right way of tokenizing?
This issue is addressed by \citeauthor{MONOLINGUAL}, where a mid-scale investigation was done to see whether different languages actually need more specific tokenizers compared to generalized tokenizers.
They report an improvement of model accuracy and F-score across all tasks and languages (\cite{MONOLINGUAL}).
While this sketches a commission for \ac{nlp} to always consider choosing a method tailored for single languages, the answer to the problem of performance versus maintenance in models might not be as elaborate as treating every language singularly.
Every language is undoubtedly unique, but that does not rule out simplification by means of further classifying and grouping target languages.
In an effort to explain the provenience and relatedness of languages many tools in the domain of typology, \ac{nlp} and indo-european studies have been constructed.

Whether it be identification of morphological features (\cite[42--56]{comrie1989}), complexity measures (\cite{measuresofMC}) or connection through reconstruction (\cite{INDOEUROPE}),
the different linguistic disciplines suggest observable regularities by which to morphologically group languages.
Leveraging the relatedness of languages in NLP is not a new idea in tokenization or \ac{pos} \hyphen tagging, but is seeing mixed results up to this day, even with augmentation methods (\cite{mixednoiseinterlanguage}).
The options seem to branch out quickly, but the mechanism of clearly separating lexemic and functional information seems inherent to most languages.
The way they differ is in they combine grammatical functions in morphemes (fusion) and bind them to lexemic morphemes (synthesis).
This may be why approaches with stemming, lemmatization or other morphological analyses are very relevant to building good tokenizers for all languages on the isolating to synthetic spectrum (\cite[51--53]{POLYSYNTHLM}).


\section{Target Languages}
\label{sec:target-languages}
This section identifies target languages that share common morphological features with German.
It is assumed that languages of the same morphological type will behave similarly when analyzed morphologically.
German was selected to serve as example language within the family of fusional languages.
The aim is not to propose yet another case study of German, but to introduce German as a surrogate to further the scope of application on other languages of similar morphological complexity.

German (ISO639-3: \texttt{deu}) is a west-germanic language and the official language of Germany, Austria, Switzerland, Liechtenstein and Luxemburg (\cite{METZLER2016}).
It is an inflectional synthetic language with approximately 130 million speakers\footnote{https://de.statista.com/statistik/daten/studie/1119851/umfrage/deutschsprachige-menschen-weltweit/ Last accessed: 09.01.23}.
German is largely researched and is still paid much attention to in the domain of (computational) linguistics.


On one side, linguistic typology has come up with many useful classifications for languages.
On the other, in the pursuit of reconstructing languages Indo-European studies have established a widely accepted phylogenetic model of the diachronic dependecy of Indo-European languages.
Both disciplines contribute to language classicifications that are used in this subsection.

Morphological complexity is a term to describe how languages use paradigms to connect grammatical information with lexemic information (\cite{MORPHOLOGICALCOMPLEXITY}).
Mind that morphological complexity is a nominal category to describe gradients of function-to-morpheme correspondence and measure of morphematic agreement, not a qualitative assessment.
The common denominators that make languages morphologically complex are their morphological features.
Those languages that use affixation, fusion, composition and derivation (among others) are all fit candidates compared to german.
A summary of morphological typology is provided in ~\cite*[78--93]{LINGUISTICTYPOLOGY}).

Due to the scope of this thesis, German will be the exemplary target language for the experimental setup.
Its morphological complexity can be compared to other related or non-related languages as shown in \autoref{tab:similarity}.
There still is no universally accepted measure the complexity of a language due to , but groupings exist on different parameters:
\\
\begin{table}[h]
    \centering
    \begin{tabular}{lll}
        \toprule
        \textbf{Language} & \textbf{Similarity} & \textbf{ISO 639--3} \\
        \midrule
        Norwegian & Closely Related & isl \\
        Danish & Closely Related & nor \\
        Dutch & Closely Related & nld \\
        English & Closely Related & eng \\
        Icelandic & Closely Related & isl \\
        Romanian & Morphology & ron \\
        Spanish & Morphology & spa \\
        Finnish & Morphology & fin \\
        Italian & Morphology & ita \\
        Hungarian & Morphology & hun \\
        \bottomrule
    \end{tabular}
    \caption{Listing of languages similar to German given the type of similarity based off \textcite{lehmann} and \textcite{MORPHOSYNTAXCOMPLEXITY1}. ISO identifiers provided at WALS\footnote{https://wals.info/languoid}.}
    \label{tab:similarity}
\end{table}

There have been interpolations between human judgements and statistical measures (on similarity of languages) which can be taken into consideration (\cite{bentz-etal-2016-comparison}).
The point to be taken is that while there is no definite proof of concept for tokenizations being effective when connecting target languages through morphological parameters, there is a strong suggestion in data and intuition of researchers that tokenization for morphologically similar languages should profit from these similarities.

With an arguable exception to English, the languages in \autoref{tab:similarity} treat their lexemes with similar morphological processes.
The upcoming interlinear glossings (as per Leipzig Glossing Rules\footnote{https://www.eva.mpg.de/lingua/pdf/Glossing-Rules.pdf}) provide examples for inflectional morphology in verbs within this group of languages.
To outline word formation processes, a glossing from \textcite[71]{finverbs} is considered:

\begin{figure}[h]
\centering
    \label{fig:fin}
    \begin{exe}
        \ex
        \gll  Tytöt istu-i-vat tuolilla \\
            girls sit-\textsc{pst}-3\textsc{pl} chair.\textsc{ade}\\
        \glt  `The girls sat on the chair.'
    \end{exe}
\end{figure}

This Finnish sentence is a textbook example of agglutinative inflectional morphology.
The verb \{istu\} is inflected by suffixing two morphemes marking the past tense \{i\} and the third person plural \{vat\} (curled brackets denote morpheme boundaries).
In this case every morpheme expresses one grammatical function, apart from \{istu\} which contains the lexical information for the verb \textquotedblleft to sit \textquotedblright.
The functional morphemes in example (1) follow the word to be inflected.
With many more functional morphemes present in Finnish, \citeauthor{finverbs} reports that there is an order in which inflectional morphemes usually appear.
In consequence, analyzing Finnish verbs results in different but reoccuring patterns depending on the degree of inflection.
The lexemic morpheme \{istu\} can be modified by \{i\} alone to just express past tense and still be productive.
Following up on the idea of agglutination, Hungarian applies an slightly different strategy to achieve inflection:

\begin{figure}[h]
\centering
\label{fig:hun}
\begin{exe}
    \ex
    \gll  Tegnap meg-hallgattunk egy lányt. \\
    yesterday \textsc{prf}-listen.\textsc{pst.1pl} a girl.\textsc{acc}\\
    \glt  `Yesterday we interviewed a girl.'
\end{exe}
\end{figure}


Hungarian is also classifed as an agglutinative language for its frequent use of affixes.
It is additionally known to combine several grammatical functions into one morphene as can be seen in example (2) as given by \textcite[262]{hunverbs}, making it a hybrid of agglutinative and fusional.
The analysis of the verb in (2) does not allow canonical segmentation to the stem although there is an underlying form \{hall;\textasteriskcentered\} meaning \textquotedblleft hear \textquotedblright.
Instead, it exists in inflection paradigms like the given example \{hallgatunk\} combinding the grammatical categories past tense and first person plural.
The morphemes addressed so far where either suffixed or not segmentable.
As lexical part \{hallgatunk\} receives a prefix \{meg\} expressing perfect tense.
This use of morphemes can also be seen in germanic or romance languages, like Italian (adapted from ~\textcite[163]{itaverbs}):

\begin{figure}[H]
\centering
    \label{fig:ita}
    \begin{exe}
        \ex
        \gll  far=se=la sotto \\
        do-\textsc{refl.prt}-\textsc{pron.prt} under\\
        \glt  `To quake in one\textquotesingle s boots’'
    \end{exe}
\end{figure}

Here the \{se\} and \{la\} carry two functions each and are suffixed to \{far\}, showing that there are morphological types in between agglutinating and fusioning.
Arguably, \{se\} being a clitic pronoun that will appear in different positions acting as indirect object, but never independent of the verb.


After a partial look on the classified languages two important empirical descriptive caveats remain: there are exceptions to almost every regularity in languages.
No language is entirely consistent in following a morphosyntactical paradigm, meaning no language is entirely fusional or agglutinative (same applies to the synthetic and isoling spectrum).
Judging from the word shapes in the data and literature, the way languages modify their stems or lexemic morphemes is largely based on affixation.
In a tokenizer acknowledging lexemic parts of words, the knowledge of word formation in the target language should be conveyed.


    \chapter{Methodology}
    \label{ch:methodology}
    in this section the whole methodoloy is covered. what do i use in this thesis, why do i use it and lastly, how?
make sure the why covers methodological implications. (vergiss nicht alle pakete als quelle im Anhang)

\section{Requirements}
\label{sec:requirements}
A series of tools will help to achieve lexicalized tokenization.
They will be explained in this chapter along with their methodological edge.

% EVTL. MACHINE LEARNING MODEL AUFSPLITTEN IN MODEL, TOKENIZER VOCABULARY?
\subsection{Machine Learning Model}
\label{subsec:mlm}

\ac{bert} is a language learning transformer model designed for \ac{nlp} tasks (\cite{ATTENTION}).
Upon release it achieved higher performance scores compared to previously used \ac{lstm} models (\cite{BERTHIGH1}).
Two main model characteristics can be observed for \ac{bert}.
Firstly, it is the first \ac{lm} to implement simultaneous attention heads, allowing for bidirectional reading.
The methodological implication of reading to the left and right of a token is to include more information about the language in single embeddings.
Secondly, \ac{bert} introduced the (at the time novel) \ac{mlm} method for training.
The method involves masking a specified amount (default 15\%) of random tokens in the input sequence.
Masked tokens are guessed by the model which can then update its weights according to success or failure.

The \ac{nlp} community has since developed \ac{bert} and adapted it to the needs of contemporary \ac{nlp} problems (roberta, germanbert, mbert \uppercase{citation}).
Its wide support, comparability and versatility make \ac{bert} the model of choice for this thesis.
Another notable feature in \uppercase{bert} is the implementation of the WordPiece tokenizer module (\uppercase{\href{https://huggingface.co/course/chapter6/6?fw=pt}{quelle?}}).
Default BERT WordPiece tokenization is predominantly heuristic by combining strings based on a precalculated score.
A variety of pre-trained tokenizers are available, although they come with a caveat.
Once a tokenizer is trained on a dataset it is specific to that dataset.
This means the application of a tokenizer on another dataset may result in out-of-vocabulary issues and different token/subtoken distributions.

Particularly relevant to this thesis is the option to train an own tokenizer from the base module.
Usually, WordPiece generates its own set of subtokens called \textit{vocabulary}.
Tokens are then \uppercase{WORDPIECE algorithmus erklären}
By providing an algorithmically generated vocabulary to WordPiece and then training it on a new dataset the tokenization behavior is changed.




\subsection{Data}
\label{subsec:data}
explain the data that is used

\subsection{Benchmark}
\label{subsec:benchmark}
explain olmpics

\section{Implementation}
\label{sec:implementation}

Tatsächliche Anwendung der Methoden auf die Daten

\subsection{Tokenizer}
\label{subsec:tokenizer}
\uppercase{essentially deriving sensible subtokens to represent lexemes}

\subsubsection{Generating a custom pre-training vocabulary}
\label{subsubsec:generating-a-custom-pre-training-vocabulary}

The Wordmap algorithm as shown in \autoref{alg:wordmap} is the first step to extracting morphemes from a token.
Its purpose is to compare two strings and store their intersections in a map of boolean values.


\algrenewcommand\algorithmicrequire{\textbf{Input:}}
\algrenewcommand\algorithmicensure{\textbf{Output:}}
\begin{algorithm}
    \caption{Wordmap generation}\label{alg:wordmap}
    \begin{algorithmic}[1]
        \Require $verbs = \{v : v \in C \wedge v_{\uppercase{POS}}\}, target$ \Comment{Set of single-\uppercase{POS} lexemic tokens}
        \Ensure $maps = (map_{1}, \ldots, map_{|verbs|})$
        \State $pair = (target, v)$
        \State $s = \textsc{shorter}(pair)$
        \State $l = \textsc{longer}(pair)$
        \State $case = \textsc{match\_ends}(pair)$ \Comment{Returns if strings match in the last or first position}
        \State $len = \textsc{len}(l)$
        \State $\delta = \Delta(len - \textsc{len}(s))$
        \\

        \Function{wordmap}{w1, w2, \*d=0}
            \State $f_{1}: c1, c2 \mapsto c1 == c2$
            \State $f_{2}: \left(\sum_{i=0+d}^{\mid w1 \mid} w1[i], w2[i] \mapsto f_{1}(w1[i], w2[i])\right)$
            \State \textbf{return} $f_{2}$
        \EndFunction
        \\
        \If{$\textit{case: any match}$}
            \If{$\delta$}
                \If{$\textit{case: left match}$}
                        \State $\textsc{wordmap}(l, s)$
                    \State Pad map from right side with 0s to match $len$
                \EndIf
                \If{$\textit{case: right match}$}
                        \State $\textsc{wordmap}(l, s, \delta)$
                    \State Pad map from left side with 0s to match $len$
                \EndIf
            \Else
                \State $\textsc{wordmap}(l, s)$
            \EndIf
        \EndIf

    \end{algorithmic}
\end{algorithm}

% DESCRIPTION OF ALGORITHM 1

Wordmap requires two \textbf{inputs} $verbs$ and $target$.
The resulting wordmap will be generated for $target$, while $verbs$ serves as comparison.
$verbs$ is a set of tokens pertaining to the same \ac{pos} category.
Note that $verbs$ should only contain those POS-tagged tokens that are expected to carry lexical information (e.g.\ verbs, adjectives, etc.).
The set is previously extracted from the corpus by \ac{pos}\hyphen tagging.
Optionally, the set can be augmented by manually adding \ac{pos} matching tokens from external sources.
The 2-tuple $pair$ are the strings to be compared.
It is passed on to (1) \textsc{shorter}, a function returning the shorter of both strings (2) \textsc{longer}, expectedly returning the longer of both strings (3) \textsc{match\_case}, a function to determine the behavior of the algorithm later on.
As two strings are compared \textsc{match\_case} captures three cases: $pair$ matches in the first, last or both positions.
Finally $len$ denotes the length of the longest string and $\delta$ difference in length between $s$ and $l$.
$\delta$ functions as an offset for index-based comparisons in \textsc{wordmap}.

\textsc{wordmap} is the function responsible for generating the wordmaps.


\begin{table}
    \centering
    \begin{tabular}{llll}
        \toprule
        \textbf{String} & \textbf{Wordmap} & \textbf{Case} & \textbf{Padding} \\
        \midrule
        \texttt{verarbeiten} & 111000000 & left match & Yes \\
         & 001000011 & right match & Yes \\
        \texttt{variiert} & 101001000 & left match & Yes \\
        \texttt{vormachen} & 101000111 & any match & No \\
        \texttt{anstrebtest} &  & no match &   \\
        (...) &  &  &  \\
        \bottomrule
    \end{tabular}
    \caption{Example wordmaps for $target = \texttt{verstehen}$}
    \label{tab:}
\end{table}

Once every $v$ has been compared to $target$, $maps$  of characters occurring in their respective positions in $target$.



    \subsubsection{Tokenizer Training}
    \label{subsubsec:tokenizer-training}
    How did I train the tokenizer, how did it go?
    Which problems arose? What went well? What happened?

    \subsection{Masked Language Model}
    \label{subsec:masked-language-model}
    Model implementation and parameters, runtimes?

    \subsection{oLMpics Benchmark}
    \label{subsec:olmpics-benchmark}



    tweak des tokenizers: segmentation ist eine frage der interpretation.
    The
    Ultimately, segmentation is a matter of interpretation.
    As mentioned in~\ref{subsec:mlm}, the default WordPiece Tokenizer lacks
    A linguistically informed



    The field of NLP  (\cite{METZLER2016}) has been expanded ever since the emergence of the language models.
    Natural language processing is understood as the


    \chapter{Results}
    \label{ch:results}
    \section{Benchmark}
\label{sec:benchmark}
How did the model Perform in the benchmark test? Report the performance for different tasks and visualize it

\section{Tokenization}
\label{sec:tokenization}
Show specific examples of tokenization and analyze the qualitatively (maybe quantitatively)


    \chapter{Discussion}
    \label{ch:discussion}
    %Method: tokenizer threshold.
Interestingly, there are many segment thresholds to choose from

there should be a filter to mitigate the natural distribution of common characters like s, n to be detected as  part of a morpheme

whole integrated system with overhead pos for all lexical categories
scaled with the degree of fusion and inflection that language has

algorithm doesnt see circumfixes like ge---t) but only affixes, this could be possible with better noise control and map interpolation

Characterizing the functional and lexical morphemes by their respective average length has had a positibe impact on segmentation, seem to have impacted the models performance very much.

it is hard to tell if the results are good without combining further metrics like loss or confusion matrix

"A low precision score (<0.5) means your classifier has a high number of False positives which can be an outcome of imbalanced class or untuned model hyperparameters. In an imbalanced class problem"
-> unbalanced sampling


    \chapter{Conclusion}
    \label{ch:conclusion}
    %What was done?
How did it go?
What went wrong?
What went well?
What was learned from this?
What are future applications?


This thesis

In almost all statistical modeling the goal is to model reality as precisely as feasible.
Language models are no exception.
The accuracy of a model should increase with the number of functional components of natural language being integrated into the model.
This is seen in e.g.\ the implementation of vocabularies, just one of many attempts to automatically identify meaningful units in language.
Even higher levels of language found in domains from ordinary pragmatics to scientific reasoning are sought after in language modeling.
While languages are observed to change slowly over time, sometimes dropping and adding features of their inventory, computational linguistics has to keep producing models that keep up with the reality of language.
The supervised tokenization in this thesis illustrates just a small part of the potential in tailored modeling.



reanalysis -> lexicalization challenges this method?
lexeme detection very close to the mean in wormaps!


encourage the formal characterization of functional and lexical morphemes


whole integrated system with overhead pos for all lexical categories
scaled with the degree of fusion and inflection that language has

use \ac{cl}
holistic

analog to the interaction hypothesis we have to stimulate models with selective input alongside generic input data to come closer
menschen lernen / sprechen /speichern sprache mit lexikon, und ML macht eine krücke über tokenizer. man sollte beides so nah aneinander bringen wie möglich


    \appendix
    \printbibliography
\end{document}
