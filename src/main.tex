% arara: xelatex: { shell: yes }
% arara: biber
% arara: nomencl
% arara: xelatex: { shell: yes }
% arara: xelatex: { shell: yes }
\documentclass[english]{ttlab-qualify}
% mgliche Optionen:
% - ngerman
% - english
% - minted
% - algorithm
% - nomencl
% - nolibertine

% P A C K A G E S

\usepackage{acronym}
\usepackage[super]{nth}
\usepackage{framed, enumitem}
\usepackage{textcomp}

\addbibresource{main.bib}


\begin{document}
    \titlehead{
        Ricardo Lukas Jung\\
        6227492\\
        Empirische Sprachwissenschaft (B.A.)\\
        Phonetik \& Digital Humanities \\
        15\textsuperscript{th} Semester\\
        s2458588@stud.uni-frankfurt.de
    }
    \subject{Thesis submitted in fulfilment of the requirements for the degree of Bachelor of Arts}
    \author{Ricardo Lukas Jung}
    \title{Lexicalizing a BERT Tokenizer}
    \subtitle{Building Open-End MLM for Morpho-Syntactically Similar Languages}
    \date{Date of Submission: \\\today}
    \publishers{Text Technology Lab\\Prof. Dr. Alexander Mehler\\Dr. Zakharia Pourtskhvanidze}

    \maketitle


    \tableofcontents

    \listoffigures
    \listoftables
    \chapter*{List of Acronyms}
    \begin{acronym}
        \acro{bert}[BERT]{Bidirectional Encoders from Transformers}
        \acro{cl}[CL]{Computational Linguistics}
        \acro{lm}[LM]{Language Model}
        \acro{lstm}[LSTM]{Long Short-Term Memory}
        \acro{ml}[ML]{Machine Learning}
        \acro{mlm}[MLM]{Masked Language Model}
        \acro{nlp}[NLP]{Natural Language Processing}

    \end{acronym}

    \chapter{Introduction}
    This thesis shows the use of specific intervention in tokenization subsystems of machine learning.


    This chapter covers the background, intentions and scope of the thesis. Explain what the thesis is about.
    \section{Motivation}
    There is an ongoing urge in the \acf{cl} community to understand natural language.
    Research in the past decades shows use of frequentist and statistical methods (such as \uppercase{ZITATION}) to their advantage, leading to the emergence of the first machine learning (ML) models.
    It became apparent that these ML models are the best currently available approach to an automated understanding of natural language.
    The structural parallels of machine learning to human learning have often been drawn (\uppercase{zitation)}) to demonstrate how similar and more importantly: how different both can be.
    A powerful feature of \ac{ml} (as opposed to human learning) is the possibility of actively controlling the the learning parameters in a supervised environment.
    To test the efficiency of \uppercase{ML} parameters a variety of tasks (\uppercase{zitation}) are designed and applied.
    A trained model will yield performance scores based on the quality of its training, much like humans on language tests.
    But the automated modeling of language is not the first instance language modelling in a broader sense.
    Traditional linguistics (\uppercase{definition} has produced fundamental research the prior to the discovery of \uppercase{ML} architectures and their implementation.
    While generic ML frameworks seem appealing in the presumption that they require less work to reach somewhat satisfactoy results,
    they are far from complete or perfect. The integration of aforementioned traditional linguistic knowledge into learning processes for machine learning is the underlying motivation of this thesis.

    Language learners usually build up a lexicon consisting of lexemes which they will have to analyze accurately in order to be productive in that target language.
    A ML model relies on a tokenizer to create such a vocabulary  (\textbf{ZITATION}).
    It is programmed to segment tokens into subwords (if possible) and provide a vocabulary comprising all the components needed to analyze a given string.
    Ideally those subwords will be part of the functional vocabulary in the target language, so called morphemes \textbf{ERKLÄRUNG}.
    A morpheme is defined as the smallest unit carrying meaning in a language.
    The morphemes of a language and its generated tokenizer vocabulary rarely coincide.
    Typically, tokanizer vocabularies will contain a lot of noise and linguistically nonsensical segmentations or words.
    Following the guiding principle that \textbf{input quality is ouput quality} not only in language learning, the morpheme vocabulary is identified as the point of leverage in the upcoming section.
    Note: explain why i use tokens and words, they are interchangeable right?
    holistic, need less attention to produce satisfactory
    
    \section{Hypotheses}
    The following research questions will be formulated for testing:
    \begin{framed}
        \begin{itemize}[itemindent=1em]
            \item[HYP1:] Adjustments to tokenization have significant impact on an \ac{lm}s performance.
        \end{itemize}
    \end{framed}
    How to achieve this hypothesis?
    \begin{framed}
        \begin{itemize}[itemindent=1em]
        \item[HYP2:] Providing lexical information to a tokenizer increases benchmark accuracy on MLM tasks.
    \end{itemize}
    \end{framed}
    How to achieve this hypothesis?

    \section{Scope and Structure}
    What is covered and what not?
    What is the shape of this thesis and what order does it have?

    \chapter{Overview}
    \section{State of the Art}
    describe the most recent findings on morphologically pretrained models in machine learning literature

    \chapter{Methodoloy}
    in this section the whole methodoloy is covered. what do i use in this thesis, why do i use it and lastly, how?
    make sure the why covers methodological implications. (vergiss nicht alle pakete als quelle im Anhang)

    \section{Requirements}
    A series of tools will help to achieve a lexicalized tokenization.
    They will be explained in this chapter along with their methodological edge.

    \subsection{Machine Learning Model}
    \ac{bert} is a language learning transformer model designed for \ac{nlp} tasks (\cite{ATTENTION}).
    Upon release it achieved higher performance scores compared to previously used \ac{lstm} models (\cite{LSTM}).
    Two main model characteristics can be observed for \ac{bert}.
    Firstly, it is the first \ac{lm} to implement simultaneous attention heads, allowing for bidirectional reading.
    The methodological implication of reading to the left and right of a token is to include more information about the language in single embeddings.
    Secondly, \ac{bert} introduced the (at the time novel) \ac{mlm} method for training.
    The method involves masking a specific amount of random tokens in the input sequence.
    Masked tokens are then guessed by the model which can then update its weights according to success or failure.

    The \ac{nlp} community has since developed \ac{bert} and adapted it to the needs of contemporary \ac{nlp} problems (roberta, germanbert, mbert \uppercase{citation}).
    Its wide support, comparability and versatility make \ac{bert} the model of choice for this thesis.
    Another notable addition is \uppercase{bert}s implementation of the WordPiece tokenizer module (\uppercase{quelle?}).



    tweak des tokenizers: segmentation ist eine frage der interpretation.
    und bidirectional heißt mehr kontext, also auch vermutlich bessere embeddings durch die identifikation der lexeme
    \subsection{Data}
    explain the data that is used
    \subsection{Benchmark}
    explain olmpics

    \section{Implementation}
    Tatsächliche Anwendung der Methoden auf die Daten

    



    The field of NLP  (\cite{METZLER2016}) has been expanded ever since the emergence of the language models.
    Natural language processing is understood as the
    \\

    cite (\cite{METZLER2016})\\
    citeast (\cite*{METZLER2016})\\

    cite (\cite{DEEPL})\\
    citeast ()\cite*{DEEPL})\\

    \begin{quote}
        This is a quote
    \end{quote}
    The intent of this thesis is to inject linguistic bias into the machine learning framework of BERT to sharpen the analytical capacities of a masked language model.
    This is done by altering the
    \appendix
    \printbibliography
\end{document}
