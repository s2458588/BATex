\section{Related Works}
\label{sec:related-works}

describe the most recent findings on morphologically pretrained models in machine learning literature
findings on POS effect on ML

The problem of captchuring lexical morphemes at the tokenization level is a problem of morphological analysis.
A standard approach implemented in the BERT architecture is WordPiece tokenization, which takes pairs of ngrams and calculates a score based on the overall frequency in the text as well as cooccurrences.
This was idea pioneered by Schuster and Nakajima (\citeyear{WORDPIECEOG}) to deal with the problem of near infinite vocabularies and was later adopted in a google research project designing the final WordPiece algorithm ~\cite{WORDPIECEGOOGLE}.
In the past, many efforts towards morphological tokenization have been made.
This thesis was mainly inspired by the FLOTA

flota and wordpiece tokenizer themselves are crude shot


\section{Target Languages}
\label{sec:target-languages}
This section identifies target languages that share common morphological features with German.
It is assumed that languages of the same morphological type will behave similarly when analyzed morphologically.
German was selected to serve as example language within the family of fusional languages.
The aim is not to propose yet another case study of German, but to introduce German as a surrogate to further the scope of application on similar languages.

Describe german (ISO639-3: deu) and its morphological state. Compare to other languages with interlinear glossing.


German (ISO639-3: \texttt{deu}) is a west-germanic language and the official language of Germany, Austria, Switzerland, Liechtenstein and Luxemburg (\cite{METZLER2016}).
It is an inflectional synthetic language with approximately 130 million speakers\footnote{https://de.statista.com/statistik/daten/studie/1119851/umfrage/deutschsprachige-menschen-weltweit/ Last accessed: 09.01.23}.
German is largely researched and is still paid much attention to in the domain of (computational) linguistics.


\subsection{Pooling similar languages}
\label{subsec:german-as-example}



On one side, linguistic typology has come up with many useful classifications for languages.
On the other, in the pursuit of reconstructing languages Indo-European studies have established a widely accepted phylogenetic model of the diachronic dependecy of Indo-European languages.
Both disciplines contribute to language classicifications that are used in this subsection.

Morphological complexity is a term to describe how languages use paradigms to connect grammatical information with lexemic information (\cite{MORPHOLOGICALCOMPLEXITY}).
Mind that morphological complexity is a nominal category to describe gradients of function-to-morpheme correspondence (\textbf{QUELLE}, not a qualitative assessment.
The common denominators that make languages morphologically complex are their morphological features.
Those languages that use affixation, fusion, composition and derivation (among others) are all fit candidates compared to german.

A summary of morphological typology is provided in ~\cite*[78--93]{LINGUISTICTYPOLOGY}).


Thus, German will be the exemplary target language for the experimental setup.

Typological findings on Indo-European languages .
and by means of composition and derivation.





Describe what morphological complexity is.

Bearman, Brown and Corbett


\begin{figure}
    \label{fig:glossing}
    \begin{exe}
        \ex
        \gll  My s Marko poexa-l-i avtobus-om v Peredelkino \\
        1\textsc{pl} \textsc{com} Marko go-\textsc{pst}-\textsc{pl} bus-\textsc{ins} \textsc{all} Peredelkino \\
        \glt  `Marko and I went to Perdelkino by bus.'
    \end{exe}
\end{figure}



Describe what similar languages exist (typological vs topological)

