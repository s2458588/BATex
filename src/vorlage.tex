% arara: xelatex: { shell: yes }
% arara: biber
% arara: nomencl
% arara: xelatex: { shell: yes }
% arara: xelatex: { shell: yes }
\documentclass[english]{ttlab-qualify}
% mgliche Optionen:
% - ngerman
% - english
% - minted
% - algorithm
% - nomencl
% - nolibertine

% P A C K A G E S
\usepackage[super]{nth}

\addbibresource{main.bib}


\begin{document}
  \titlehead{
    Ricardo Lukas Jung\\
    6227492\\
    Bachelor\\
    Empirische Sprachwissenschaft (Phonetik \& Digital Humanities) \\
    \nth{15} Semester\\
    s2458588@stud.uni-frankfurt.de
  }
  \subject{Thesis submitted in fulfilment of the requirements for the degree of Bachelor of Arts}
  \author{Ricardo Lukas Jung}
  \title{Lexicalizing a BERT Tokenizer}
  \subtitle{Building Open-End MLM for Morpho-Syntactically Similar Languages}
  \date{Date of Submission: \\\today}
  \publishers{Text Technology Lab\\Prof. Dr. Alexander Mehler\\Dr. Zakharia Pourtskhvanidze}

  \maketitle


  \tableofcontents

  \chapter{Introduction}
  \label{ch:vorlageintroduction}
  The field of NLP \textit{natural language processing} (\cite{METZLER2016}) has been expanded ever since the emergence of the language models.
  Natural language processing is understood as the
  \\

  cite (\cite{METZLER2016})\\
  citeast (\cite*{METZLER2016})\\

  cite (\cite{ONLINETEST})\\
  citeast \cite*{ONLINETEST})\\
  The intent of this thesis is to inject linguistic bias into the machine learning framework of BERT to sharpen the analytical capacities of a masked language model.
  This is done by altering the
  \appendix
  \printbibliography
\end{document}
