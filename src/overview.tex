\section{State of the Art}
\label{sec:state-of-the-art}
describe the most recent findings on morphologically pretrained models in machine learning literature

\section{Target Languages}
\label{sec:target-languages}
Describe german (ISO639-3: deu) and its morphological state.

German (ISO639-3: deu) will be the exemplary target language for the experimental setup.
German is an inflectional language making use of composition and derivation.
A morpheme is defined as the \textquote{smallest distinctive unit of a language having a definite grammatical function} (\cite{morpheme}).
It is a west-germanic language and the official language of Germany, Austria, Switzerland, Liechtenstein and Luxemburg (\cite{METZLER2016}).
Approximately 130 million speakers\footnote{https://de.statista.com/statistik/daten/studie/1119851/umfrage/deutschsprachige-menschen-weltweit/}
German can be defined as fusional on morphemic level and agglutivation


Describe what morphological complexity is.
Morphological complexity is a term to describe how languages use paradigms to connect grammatical information with lexemic information (\cite{MORPHOLOGICALCOMPLEXITY}).
Bearman, Brown and Corbett
Morphological complexity is a nominal category to describe gradients of function-to-morpheme correspondence.
\cite{ATTENTION}
\begin{exe}
    \ex
    \gll  My s Marko poexa-l-i avtobus-om v Peredelkino \\
    1\textsc{pl} \textsc{com} Marko go-\textsc{pst}-\textsc{pl} bus-\textsc{ins} \textsc{all} Peredelkino \\
    \glt  `Marko and I went to Perdelkino by bus.'
\end{exe}



Describe what similar languages exist (typological vs topological)
